\begin{titlepage}
    \noindent
    University of Eastern Finland, Faculty of Science, Forestry and Technology\\
    Master's programme in Technical Physics\\
    Niilo Saarlemo: Implementation of single photon emission computed tomography (SPECT) image reconstruction in OMEGA software\\
    Master's thesis, 57 pages, 2 appendices (10 pages)\\
    Supervisors: PhD Ville-Veikko Wettenhovi, PhD Matti Kortelainen\\
    September 2024\\

    \hrule

    \vspace{4pt}\noindent
    Keywords: single photon emission computed tomography, SPECT, SPET, reconstruction, nuclear medicine\\
    \vspace{24pt}\\
    \noindent
    Single-photon emission computed tomography (SPECT) is a nuclear medicine imaging technique used to determine the distribution of gamma-emitting radiotracers within the body. This method is widely used for diagnosing cardiovascular diseases, stroke, infections and cancer. In the SPECT imaging process, the actual distribution of radioactivity is not measured directly. Instead, the detected gamma photons are binned into two-dimensional histograms called projections from different directions around the body. Thus, the actual image of the distribution of the radioactivity has to be reconstructed from the measurements.
    
    When considering the distribution of radioactivity as piecewise constant in the voxels of the imaging area, the image reconstruction is reduced into solving a matrix equation $Af=g$, where $A$ is the projection matrix, into which the geometry of the measurement is embedded, $f$ is the image to be determined and $g$ consists of the measured projections.
    
    There are multiple methods for forming the projection matrix $A$. So-called rotation-based methods consider the radiation detectors to be fixed and the imaging area to rotate. In ray-based methods, intersection of the voxels and the rays along which the detected gamma photons have travelled are considered.
    
    OMEGA is a piece of open-source software intended for tomographic image reconstruction. Currently, in the scope of SPECT, OMEGA has support for only rotation-based methods for forming the projection matrix. As these rotation-based methods may be difficult to implement for novel detector geometries, this thesis presents a model for a more flexible ray-based projector.
    
    \newpage\thispagestyle{empty}

    Validation of the model was done using projections measured from one phantom and simulated from two phantoms. In addition to visual inspection, the reconstructions were analyzed quantitatively using the structural similarity index measure (SSIM) and mean square error (MSE). The developed ray-based model turned out to match the rotation-based projector in quality of the reconstructed image while being faster to compute. This makes the ray-based model a viable alternative to the already supported rotation-based projector. However, further validation with extensive clinical datasets is necessary to confirm the model's reliability.
    
    The thesis highlights the potential for future enhancements, such as the application of appropriate prior information about the image to be reconstructed, parallel computing using graphics processors, and distribution of the rays in the model.
\end{titlepage}