\begin{titlepage}
    \section*{Abstract}
    Single-photon emission computed tomography (SPECT) is a medical imaging technique used to determine the distribution of gamma-emitting radiotracers within the body. This method is widely used for diagnosing cardiovascular diseases, stroke, infections and cancer. Unlike X-ray computed tomography (CT), which visualizes anatomical structures, SPECT provides insights into the physiological functions of tissues and organs. Central to the performance of SPECT is the collimator --- a thick lead sheet with regularly arranged holes --- that filters the detected photon by direction of propagation, significantly impacting the system's resolution and sensitivity. Accurate modeling of the collimator is therefore essential to consider during the image reconstruction process.

    This thesis presents the development of a SPECT image reconstruction module for the open-source OMEGA software, initially designed for positron emission tomography (PET) image reconstruction. The integration of the SPECT reconstruction module into OMEGA aims to provide a versatile and accessible tool for medical imaging research and clinical applications. The primary objective was to implement a collimator modeling algorithm that can be executed on a computer's central processing unit (CPU) and is compatible with hexagonal and square collimator hole designs in SPECT devices. The reconstruction process involves determining the paths of gamma photons through a three-dimensional space to form an image. The implementation of SPECT image reconstruction can leverage the most common reconstruction methods built-in in OMEGA.

    Testing with simulated data showed promising results, but further validation with clinical datasets is necessary to confirm the model's reliability. Challenges such as the modeling of collimator geometry and the computational efficiency of the reconstruction algorithm are discussed. The thesis highlights the potential for future enhancements, including the use of graphics processing units (GPUs) and OpenCL compatibility to accelerate computations.
\end{titlepage}