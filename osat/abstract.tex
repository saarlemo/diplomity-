\begin{titlepage}
    \section*{Abstract}
    Single-photon emission computed tomography (SPECT) is a nuclear medicine imaging technique used to determine the distribution of gamma-emitting radiotracers within the body. This method is widely used for diagnosing cardiovascular diseases, stroke, infections and cancer.
    
    In the SPECT imaging process, the actual distribution of radioactivity is not measured directly. Instead, the detected gamma photons are binned into images called projections from different orientations around the body. Thus, the actual image of the distribution of the radioactivity has to be reconstructed from the measurements.
    
    When considering the distribution of radioactivity as piecewise constant in the imaging area, the image reconstruction is reduced into solving a matrix equation $Af=g$, where $A$ is the projection matrix, into which the geometry of the measurement is embedded, $f$ is the image to be determined and $g$ consists of the measured projections.
    
    There are multiple methods for forming the projection matrix $A$. So-called rotation-based methods consider the radiation detectors to be fixed and the imaging area to rotate. In ray-based methods, rays along which the detected gamma photons have travelled are considered.
    
    OMEGA is a piece of open-source software intended for tomographic image reconstruction. Currently, in the scope of SPECT, OMEGA has support for only rotation-based methods for forming the projection matrix. As these rotation-based methods may be difficult to implement for novel detector geometries, this thesis presents a model for determining the rays required for ray-based methods. 
    
    Testing with simulated data showed promising results, but further validation with clinical datasets is necessary to confirm the model's reliability. Challenges such as the accurate modeling of the detector geometry and the computational efficiency of the reconstruction algorithm are discussed. The thesis highlights the potential for future enhancements, including the application of appropriate prior information about the image to be reconstructed.
\end{titlepage}