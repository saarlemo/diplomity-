\section{Johdanto}
Yksifotoniemissiotomografia (SPECT, \textit{single photon emission computed tomography}) on lääketieteellinen kuvantamismenetelmä, jolla määritetään gamma-aktiivisen merkkiaineen aktiivisuusjakauma kehossa\cite{bruyant_analytic_2002}. Kuvantamismenetelmää käytetään esimerkiksi sydän- ja verisuonitautien, aivohalvauksen, infektioiden ja syövän diagnosoinnissa.\cite{cherry_single_2012, van_audenhaege_review_2015, crisan_radiopharmaceuticals_2022} Transmissiotomografiasta, jossa säteilyn lähde on kehon ulkopuolella\cite{bercovich_medical_2018}, ja erityisesti tietokonetomografiasta (TT) poiketen SPECT soveltuu elinten ja kudosten fysiologisen toiminnan seuraamiseen eikä pelkästään anatomisen rakenteen kuvantamiseen\cite{bercovich_medical_2018, crisan_radiopharmaceuticals_2022, cherry_single_2012}. Nykyaikaisissa SPECT/TT-laitteissa on yhdistetty rakenne ja fysiologinen toiminta kuvattavaksi yhtäaikaisesti\cite{hutton_origins_2014, cherry_single_2012, bercovich_medical_2018}.

SPECT on suhteellisen turvallinen ja ei-invasiivinen menetelmä, joka mahdollistaa potilaiden tutkimisen vähäisin komplikaatioriskein. Toisaalta SPECT väistämättä altistaa kuvattavan potilaan ionisoivalle säteilylle, mutta kuvantamisen hyödyt ylittävät usein sen haitat. Verrattuna positroniemissiotomografiaan (PET), yksifotoniemissiotomografian resoluutio on merkittävästi huonompi, mutta menetelmä on edullisempi\cite{crisan_radiopharmaceuticals_2022, cherry_single_2012}. Esimerkiksi PET-kuvantamiseen tarvittavien radionuklidien tuottaminen vaatii syklotronin, mutta SPECT-kuvantamisessa käytetyt radionuklidit voidaan tuottaa generaattorilla. SPECT-kuvantamisessa käytetyillä radionuklideilla on myös pidempi puoliintumisaika, jonka vuoksi radionuklidien käsittely on helpompaa.\cite{crisan_radiopharmaceuticals_2022}

Kuvan rekonstruktio on olennainen osa SPECT-kuvantamisprosessia, sillä se vaikuttaa eniten kuvan laatuun ja siten diagnostiikan tarkkuuteen. Rekonstruktion aikana kerätty raakadata muunnetaan kuviksi, jotka esittävät radioaktiivisuuden jakaumaa kuvattavassa kohteessa. Rekonstruktiossa oletetaan, että aktiivisuusjakauma on paloittain vakio vokseleiksi kutsutuissa alueissa kohteen sisällä. Useammasta eri suunnasta havaitusta säteilystä pyritään määrittämään, kuinka monta radioaktiivista hajoamista on tapahtunut kussakin vokselissa tietyllä aikavälillä.\cite{bruyant_analytic_2002}

\subsection{Aihepiirin lyhyt historia}
Lääketieteen ala mullistui loppuvuodesta 1895, kun röntgensäteily löydettiin osittain vahingossa Wilhelm Röntgenin tutkiessa sähkövirran kulkua tyhjiöputkessa. Jo kuukausi säteilyn löytymisen jälkeen ionisoivaa säteilyä hyödyntäen otettiin ensimmäinen röntgenkuva ihmisestä, ja vuosisadan vaihteessa kuvantamismenetelmä oli laajalti käytössä ympäri maailmaa.\cite{bercovich_medical_2018, cherry_basic_2012} Vuotta myöhemmin Henri Becquerel havaitsi, että uraani lähettää samankaltaista säteilyä kuin Röntgenin kokeessa. Näistä tapahtumista ajatellaan saaneen alkunsa ymmärrys ionisoivan säteilyn luonteesta.\cite{cherry_basic_2012}

Hieman alle kolmeakymmentä vuotta myöhemmin syntyi isotooppilääketieteen kuvantaminen, jossa potilaalle annettavan radioaktiivisen merkkiaineen säteily havaitaan kehon ulkopuolelta. Säteilyn havaitsemisessa käytettiin tuohon aikaan Geiger-mittaria, joka asetettiin kuvattavan kohde-elimen kohdalle kehon ulkopuolelle. Vuonna 1950 säteilyn havaitsemisessa oli siirrytty tuikeilmaisimen käyttöön. Samana vuonna Benedict Cassen kehitti automatisoidun, tulostimeen kytketyn tuikeilmaisimen. Tuikeilmaisin liikkui sähkömoottoreiden avulla jäykän levyn päällä, jossa oli ruudukko. Levy asetettiin esimerkiksi kilpirauhasen kohdalle, ja yksi ruutu kerrallaan laite väritti ruutupaperia riippuen siitä, kuinka paljon säteilyä havaittiin. Näin saatiin muodostettua suhteellisen tehokkaasti ja nopeasti kaksiulotteinen kuva radioisotoopin jakaumasta.\cite{jaszczak_early_2006, cherry_gamma_2012}

Tähän asti kuvantamisessa oli keskitytty yksinomaan kaksiulotteisen projektion muodostukseen kolmiulotteisesta aktiivisuusjakaumasta tai säteilyn vaimenemisesta. Matemaattisten menetelmien ja laskentatekniikan kehitys avasi uusia mahdollisuuksia kuvantamiseen, kun kaksiulotteisista projektioista pystyttiin määrittämään kolmiulotteinen aktiivisuusjakauma\cite{bercovich_medical_2018, cherry_single_2012} tai säteilyn vaimenemiskartta\cite{bercovich_medical_2018, stiller_basics_2018}. Kolmiulotteisen vaimenemiskartan kautta voidaan tutkia tarkemmin erityisesti ihmiskehon rakennetta ja aktiivisuusjakauman avulla fysiologista toimintaa.

Ensimmäinen nykyaikainen, analoginen gammakamera saatiin toimintaan vuonna 1958\cite{hutton_origins_2014, cherry_gamma_2012}. Vuonna 1963 David Kuhl tutkimusryhmineen käynnisti emissiotomografialaitteen suunnittelun, josta muotoutui lopulta perusta nykyaikaisille laitteille\cite{jaszczak_early_2006, hutton_origins_2014}. Emissiotomografiassa oli ongelmana aktiivuusjakauman anatominen sijoittuminen. Tästä inspiroituneena Kuhl ideoi myös transmissiokuvauksen yhdistämistä emissiokuvaukseen vuonna 1966\cite{jaszczak_early_2006}.

Yksifotoniemissiotomografian historia ulottuu vuoteen 1968, jolloin suoritettiin ensimmäinen SPECT-kuvaus. Laitteisto poikkesi merkittävästi nykyaikaisesta: laite oli kiinteästi paikoillaan kuvattavan potilaan pyöriessä tuolilla detektorin edessä.\cite{jaszczak_early_2006, hutton_origins_2014}
%
%Hieman ennen 1970-luvun puoltaväliä suoritettiin ensimmäiset kliiniset kokeet tietokonetomografialaitteella\cite{bercovich_medical_2018} (TT) ja kahden vuoden kuluttua siitä markkinoilla oli toimiva TT-laite\cite{hutton_origins_2014, willemink_evolution_2019}.

Modernin SPECT-laitteen, jossa gammakameran detektorit pyörivät kuvattavan kohteen ympärillä, visioivat Ronald Jaszczak ja John Keyes vuonna 1976, mutta kumpikin omatoimisesti\cite{jaszczak_early_2006}. Alkujaan detektoreja oli vain yksi, mutta Jaszczak kehitti usean detektorin laitteen kaksi vuotta myöhemmin\cite{jaszczak_early_2006, hutton_origins_2014}.

1980-luvulle tultaessa laitteiston tekniikka kehittyi merkittävästi datan käsittelyn rinnalla. Näin ollen kuvantamismenetelmistä tuli tehokkaampia ja nopeampia, joka vaikutti myönteisesti niiden hyödyntämiseen kliinisessä käytössä\cite{jaszczak_early_2006}. Samoihin aikoihin kehitettiin erikoistunutta emissiotomografialaitteistoa esimerkiksi aivojen ja sydämen kuvantamiseen\cite{hutton_origins_2014}.

Vaikka emissiotomografiaa ja transmissiotomografiaa yhdisteltiin jo 1960-luvulla ja tietokonetomografialaitteet (TT) olivat markkinoilla jo 1970-luvun puolessavälissä, nykyisen kaltaisia SPECT/TT-laitteita ei ollut ennen 1990-lukua\cite{bercovich_medical_2018, hutton_origins_2014, willemink_evolution_2019}. Vuosikymmenen alussa alkanut tutkimus ja kehitys johti ensimmäisen kaupallisen SPECT/TT-laitteen lanseeraamiseen vuonna 1999. Kuitenkaan ne eivät yleistyneet kliinisessä käytössä kovin nopeasti.\cite{hutton_origins_2014}

Uusimmissa SPECT/TT-laitteissa käytetään puolijohdeilmaisimia, joissa havaittu säteily muuttuu suoraan mitattavaksi sähkövirraksi. Kaupalliset puolijohdeilmaisinta käyttävät SPECT/TT-laitteet julkistettiin vuosina 2009--2010.\cite{hutton_origins_2014}.

Alkujaan emissiotomografiakuvan muodostuksessa käytettiin kuvaa askel kerrallaan tarkentavia iteratiivisia algoritmeja, mutta ne osoittautuivat liian vaativiksi silloiselle teknologialle. Tämän vuoksi 1980-luvulle asti käytettiin analyyttisia algoritmeja, jotka ovat nopeita mutta joihin on hankala sisällyttää ennalta tunnettua tietoa aktiivisuusjakaumasta.\cite{willemink_evolution_2019} Tietokonetomografiassa kuvan koko on suurempi, joten laskentatehon vuoksi iteratiiviset algoritmit otettiin kliiniseen käyttöön vasta vuonna 2009. Sittemmin rekonstruktioalgoritmien kehitys on painottunut rinnakkaislaskennan ja näytönohjainteknologian kehityksen myötä iteratiivisiin algoritmeihin.\cite{willemink_evolution_2019} 

\subsection{Työn tavoite}
Mooren lain ydinajatus on, että transistorien määrä mikroprosessoreissa kaksinkertaistuu joka toinen vuosi. Transistorien määrä on suoraan verrannollinen laitteen laskentatehoon, joten todennäköisesti sillä laitteella, jolla luet tätä diplomityötä, voisi rekonstruoida standardikokoisen SPECT-kuvan kohtuullisessa ajassa. Erityisesti on olemassa ohjelmointikielien kirjastoja ja rajapintoja, joiden avulla yhtä ohjelmistoa on mahdollista suorittaa lähes millä tahansa laitteella.

Yksi näistä ohjelmistoista on OMEGA (\textit{Open-source multi-dimensional tomographic reconstruction software}), joka on suunniteltu emissio- ja transmissiotomografiakuvan rekonstruktioon. Nykyisellään OMEGA tukee positroniemissiotomografia- (PET, \textit{positron emission tomography}) ja tietokonetomografiakuvan rekonstruktiota sekä keskusyksiköllä että näytönohjaimella OpenCL-rajapinnan avulla.\cite{wettenhovi_omegaopen-source_2021}. Toissijaisesti OMEGA on tarkoitettu alustaksi, johon on suoraviivaista toteuttaa uusia rekonstruktiomenetelmiä. 

Tämän teknillisen fysiikan diplomityön tavoitteena on toteuttaa OMEGA-ohjelmistoon tietokoneen keskusyksiköllä (CPU) ajettava SPECT-kuvan rekonstruktio. Ideana on määrittää kolmiulotteisesta avaruudesta suoria, joita pitkin kulkevia gammasäteilyn fotoneja havaitaan SPECT-laitteella. Rekonstruktio ei rajoitu yksittäiseen laitteeseen, vaan sen on tarkoitus toimia kaikissa SPECT-laitteissa joiden kollimaattorissa on kuusi- tai nelikulmaiset reiät.