\section{Johdanto}
\subsection{Aihepiirin lyhyt historia}
(1895 Röntgen) Säteilyn löytäminen\cite{bercovich_medical_2018}

Vuosisadan vaihtuessa röntgenkuvat olivat laajalti käytössä, muutti lääketieteen alan suuntaa\cite{bercovich_medical_2018}

(1928 Geiger ja Walther) Geiger-mittarin käyttö kuvantamisessa, asetettiin kuvannettavan elimen päälle.\cite{jaszczak_early_2006}

(1950--1951 Cassen) Automatisoitu tulostimeen kytketty tuikeilmaisin. Kuvannettavan alueen päälle asetettiin jäykkä levy jossa on ruudukko. Tuikeilmaisin mittasi havaitut hajoamistapahtumat jokaisesta ruudusta muodostaen kaksiulotteisen kuvan radioisotoopin jakaumasta.\cite{jaszczak_early_2006}

Tähän asti kuvantaminen oli kolmiulotteisen aktiivisuusjakauman projisointia kahteen ulottuvuuteen. Tomografiassa kaksiulotteisista projektioista yritetään määrittää aktiivisuusjakauma kolmessa ulottuvuudessa.

(1958, 1964 Anger) Nykyaikainen analoginen gammakamera.\cite{jaszczak_early_2006, hutton_origins_2014}

(1963 Kuhl ja Edwards) Ensimmäisen SPECT-laitteen suunnittelu\cite{jaszczak_early_2006, hutton_origins_2014}

(1963--1975 Kuhl) SPECT-laitteen kehitys\cite{jaszczak_early_2006, hutton_origins_2014}

(1966 Kuhl, Hale ja Eaton) Transmissiokuvaus gammasäteilyllä\cite{jaszczak_early_2006, hutton_origins_2014}

(1968 Muehllehner) SPECT-kuvaus, jossa kuvattava henkilö pyöri tuolilla ja gammakamera oli paikoillaan\cite{jaszczak_early_2006, hutton_origins_2014}

(1973 Hounsfield ja Ambrose) Ensimmäiset kliiniset kokeet CT-laitteella\cite{hutton_origins_2014, bercovich_medical_2018}

(1975) Ensimmäinen kaupallinen CT-laite\cite{hutton_origins_2014}

(1976 Jaszczak) ja (1976 Keyes) kehittivät SPECT-laitetta, jossa detektori pyöri kuvattavan henkilön ympärillä.\cite{jaszczak_early_2006, hutton_origins_2014}

(1976--1978 Jaszczak) SPECT-laite, jossa on useampi pyörivä detektori.\cite{jaszczak_early_2006, hutton_origins_2014}

1980-luvulla laitteiston tekniikka ja datan käsittely kehittyi, laitteistosta tuli tehokkaampaa\cite{jaszczak_early_2006}

1980-luvulla laitteisto erikoistui esimerkiksi aivojen ja sydämen kuvantamiseen\cite{hutton_origins_2014}

(1990 Hasegawa), (1992 Lang) SPECT/CT laitteen kehitys\cite{hutton_origins_2014}

(1999) Ensimmäinen kaupallinen SPECT/CT-laite, eivät yleistyneet kliinisessä käytössä kovin nopeasti\cite{hutton_origins_2014}

(2009--2010) kaupallinen puolijohdeilmaisinta käyttävä SPECT-laite sydämen kuvantamiseen\cite{hutton_origins_2014}
 
\textcolor{red}{Tähän säteilyä käyttävien kuvantamismenetelmien historia 1--2 sivua: CT, PET, SPECT}
\subsection{Työn tavoite}