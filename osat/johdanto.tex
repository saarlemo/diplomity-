\section{Johdanto}
\subsection{Aihepiirin lyhyt historia}
Lääketieteen ala mullistui loppuvuodesta 1895, kun röntgensäteily löydettiin osittain vahingossa Wilhelm Röntgenin tutkiessa sähkövirran kulkua tyhjiöputkessa. Jo kuukausi säteilyn löytämisen jälkeen ionisoivaa säteilyä hyödyntäen otettiin ensimmäinen röntgenkuva ihmisestä, ja vuosisadan vaihteessa kuvantamismenetelmä oli laajalti käytössä ympäri maailmaa.\cite{bercovich_medical_2018} Vuotta myöhemmin Henri Becquerel havaitsi, että uraani lähettää samankaltaista säteilyä kuin Röntgenin kokeessa. Näistä tapahtumista ajatellaan saaneen alkunsa ionisoivan säteilyn ymmärrys.

Hieman alle kolmeakymmentä vuotta myöhemmin syntyi isotooppilääketieteen kuvantaminen, jossa potilaalle annettavan radioaktiivisen merkkiaineen säteily havaitaan kehon ulkopuolelta. Säteilyn havaitsemisessa käytettiin tuohon aikaan Geiger-mittaria, joka asetettiin kuvattavan kohde-elimen kohdalle kehon ulkopuolelle. Vuonna 1950 säteilyn havaitsemisessa oli siirrytty tuikeilmaisimen käyttöön. Samana vuonna Benedict Cassen kehitti automatisoidun, tulostimeen kytketyn tuikeilmaisimen. Tuikeilmaisin liikkui sähkömoottoreiden avulla jäykän levyn päällä, jossa oli ruudukko. Levy asetettiin esimerkiksi kilpirauhasen kohdalle, ja yksi ruutu kerrallaan laite väritti ruutupaperia riippuen siitä, kuinka paljon säteilyä havaittiin. Näin saatiin muodostettua suhteellisen tehokkaasti ja nopeasti kaksiulotteinen kuva radioisotoopin jakaumasta.\cite{jaszczak_early_2006}

Tähän asti kuvantamisessa oli keskitytty yksinomaan kaksiulotteisen projektion muodostukseen kolmiulotteisesta aktiivisuusjakaumasta. Matemaattisten menetelmien ja laskentatekniikan kehitys avasi uusia mahdollisuuksia kuvantamiseen, kun kaksiulotteisista projektioista pystyttiin määrittämään kolmiulotteinen aktiivisuusjakauma. Kolmiulotteisen aktiivisuusjakauman kautta voitiin tutkia tarkemmin erityisesti ihmiskehon rakennetta ja toimintaa.

Ensimmäinen nykyaikainen, analoginen gammakamera saatiin toimintaan vuonna 1958\cite{hutton_origins_2014}. Vuonna 1963 David Kuhl tutkimusryhmineen käynnisti emissiotomografialaitteen suunnittelun, josta muotoutui lopulta perusta nykyaikaisille laitteille\cite{jaszczak_early_2006, hutton_origins_2014}. Emissiotomografiassa oli ongelmana aktiivuusjakauman anatominen sijoittuminen. Tästä inspiroituneena Kuhl ideoi myös transmissiokuvauksen yhdistämistä emissiokuvaukseen vuonna 1966\cite{jaszczak_early_2006}.

Yksifotoniemissiotomografian (SPECT, \textit{single photon emission computed tomography}) historia ulottuu vuoteen 1968, jolloin suoritettiin ensimmäinen SPECT-kuvaus. Laitteisto poikkesi merkittävästi nykyaikaisesta: laite oli kiinteästi paikoillaan kuvattavan potilaan pyöriessä tuolilla detektorin edessä.\cite{jaszczak_early_2006, hutton_origins_2014}

Hieman ennen 1970-luvun puoltaväliä suoritettiin ensimmäiset kliiniset kokeet tietokonetomografialaitteella\cite{bercovich_medical_2018} (TT) ja kahden vuoden kuluttua siitä markkinoilla oli toimiva TT-laite\cite{hutton_origins_2014, willemink_evolution_2019}.

Modernin SPECT-laitteen, jossa gammakamerat pyörivät kuvattavan kohteen ympärillä, visioivat Ronald Jaszczak ja John Keyes vuonna 1976, mutta kumpikin omatoimisesti\cite{jaszczak_early_2006}. Alkujaan gammakameroita oli vain yksi, mutta Jaszczak kehitti usean paneelin laitteen kaksi vuotta myöhemmin\cite{jaszczak_early_2006, hutton_origins_2014}.

1980-luvulle tultaessa laitteiston tekniikka kehittyi merkittävästi datan käsittelyn rinnalla. Näin ollen kuvantamismenetelmistä tuli tehokkaampia ja nopeampia vaikuttaen myönteisesti niiden hyödyntämiseen kliinisessä käytössä\cite{jaszczak_early_2006}. Samoihin aikoihin kehitettiin erikoistunutta emissiotomografialaitteistoa esimerkiksi aivojen ja sydämen kuvantamiseen\cite{hutton_origins_2014}.

Vaikka emissiotomografiaa ja transmissiotomografiaa yhdisteltiin jo 1960-luvulla, nykyisen kaltaisia SPECT/TT-laitteita ei ollut ennen 1990-lukua. Vuosikymmenen alussa alkanut tutkimus ja kehitys johti ensimmäisen kaupallisen SPECT/TT-laitteen lanseeraamiseen vuonna 1999. Kuitenkaan ne eivät yleistyneet kliinisessä käytössä kovin nopeasti.\cite{hutton_origins_2014}

Uusimmissa SPECT/TT-laitteissa käytetään puolijohdeilmaisimia, joissa havaittu säteily muuttuu suoraan mitattavaksi sähkövirraksi. Kaupalliset puolijohdeilmaisinta käyttävät SPECT/TT-laitteet julkistettiin vuosina 2009--2010.\cite{hutton_origins_2014}.

Vuoteen 2009 asti kolmiulotteisen kuvan rekonstruktio projektioista oli tehty analyyttisilla algoritmeilla, joihin on hankala sisällyttää tunnettua tietoa aktiivisuusjakaumasta. Analyyttiset algoritmit olivat kuitenkin verrattain nopeita tuolloisella teknologialla. Rinnakkaislaskenta ja uusi näytönohjainteknologia nopeuttivat iteratiivisia rekonstruktiomenetelmiä, joista ensimmäiset hyväksyttiin kliiniseen käyttöön vuonna 2009.\cite{willemink_evolution_2019}

\subsection{Työn tavoite}
Mooren lain ydinajatus on, että transistorien määrä mikroprosessoreissa kaksinkertaistuu joka toinen vuosi. Transistorien määrä on suoraan verrannollinen laitteen laskentatehoon, joten todennäköisesti sillä laitteella, jolla luet tätä diplomityötä, voisi rekonstruoida standardikokoisen SPECT-kuvan kohtuullisessa ajassa.

OMEGA (\textit{Open-source MATLAB/GNU Octave Emission and Transmission Tomography Software}) on avoimen lähdekoodin ohjelmisto, joka on suunniteltu emissio- ja transmissiotomografiakuvan rekonstruktioon. Nykyisellään OMEGA tukee positroniemissiotomografia- (PET, \textit{positron emission tomography}) ja tietokonetomografiakuvan rekonstruktiota sekä keskusyksiköllä että näytönohjaimella.\cite{wettenhovi_omegaopen-source_2021}. Toissijaisesti OMEGA on tarkoitettu alustaksi, johon on suoraviivaista toteuttaa uusia rekonstruktiomenetelmiä. 

Tämän teknillisen fysiikan diplomityön tavoitteena on toteuttaa OMEGA-ohjelmistoon keskusyksiköllä ajettava rekonstruktio SPECT-kuvalle. Rekonstruktio ei rajoitu yksittäiseen laitteeseen, vaan toimii kaikissa SPECT-laitteissa joiden kollimaattorissa on kuusikulmaiset reiät.