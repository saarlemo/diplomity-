\section{Tomografiakuvan muodostus}
Gammakameran kollimaattorin voidaan olettaa havaitsevan ainoastaan yhdestä suunnasta saapuvaa säteilyä. Tällöin gammakameran perspektiivistä havaitaan ikään kuin kaksiulotteinen varjo radioaktiivisen merkkiaineen kolmiulotteisesta jakaumasta. Keräämällä näitä projektioiksi kutsuttuja varjoja useasta suunnasta, voidaan matemaattisin menetelmin määrittää radioaktiivisen merkkiaineen jakauma.

Projektioiden matematiikka esitetään kahdessa ulottuvuudessa (jolloin projektiot ovat yksiulotteisia) ymmärrettävyyden vuoksi. Yleistys kolmeen ulottuvuuteen ja kaksiulotteiseen projektioon on suhteellisen suoraviivainen.

\begin{figure}[H]
    \centering
    \captionsetup{width=.9\textwidth}
    \begin{tikzpicture}
    \draw[->, thick] (-3,0)--(3,0) node[right]{$x$};
    \draw[->, thick] (0,-3)--(0,3) node[above]{$y$};


    \begin{scope}[shift={(0, 0)}, rotate=30]
        \begin{scope}[shift={(0, 5)}, rotate=180] % Detektori
            \draw[->, thick] (3.5, -1.3) -- (-3.5, -1.3) node[right]{$s$}; % s-akseli
            \draw[domain=-1.5:1.5, smooth, variable=\s, red] plot ({\s}, {-2.3-cos(2 * pi / 3 * \s r)}) node[above left]{$g(s, \theta)$};

            % Kollimaattori
            \foreach \x in {-3.5, -3.2, ..., 3.5}
            \fill (\x,0) rectangle (\x + 0.1,1);
    
            % Tuikeaine
            \draw (-3.5, 0) rectangle (3.5, -1);
            \fill[pattern=north east lines] (-3.5, 0) rectangle (3.5, -1);
        \end{scope}
        % Aktiivisuusjakauma
        \fill[pattern=dots, pattern color=blue] (0,0) circle (1.5) node[below left]{$f(x, y)$};

        % Suora
        \draw[dashed, red] (0.9, -3) -- (0.9, 5);
    \end{scope}
\end{tikzpicture}
    \caption{Aktiivisuusjakauma $f$ $xy$-tasossa (siniset pisteet), gammakamera (mustat palkit ja väritetty alue) ja projektio $g(s, \theta)$, jossa $\theta$ on $s$- ja $x$-akseleiden välinen kulma. Yhteen projektion pisteeseen vaikuttava aktiivisuusjakauman osa on havainnollistettu punaisella katkoviivalla. Kuva on muokattu lähteestä \cite{bruyant_analytic_2002}.}
    \label{fig:projektio}
\end{figure}

Tarkastellaan $xy$-tason jatkuvaa aktiivisuusjakaumaa $f\colon\RR^{2}\to\RR$. Olkoon $\theta$ gammakameran tangentin ja positiivisen $x$-akselin välinen kulma kuten \hyperref[fig:projektio]{kuvassa \ref*{fig:projektio}}. Merkitään projektiota $g(s, \theta)$, jossa $s$ on gammakameran tangentin suuntainen koordinaattiakseli.

Kulmalla $\theta$ ja pisteessä $s$ gammakamera havaitsee säteilyä $xy$-tason suoralta $\Omega$, jonka määrittää
\begin{equation*}
    \Omega=\left\{ (x, y) \mid x\cos(\theta)+y\cos(\theta)=s \right\}.
\end{equation*}
Projektio, eli suoralta $\Omega$ havaittu säteily voidaan esittää siten Radon-muunnokseksi kutsuttuna polkuintegraalina\cite{radon_determination_1986, bruyant_analytic_2002}
\begin{equation}\label{eqn:radon-muunnos}
    g(s, \theta)=\int_{\Omega}f(x, y).
\end{equation}

Tietokoneella voidaan kuitenkin käsitellä ainoastaan äärellistä määrää mitattuja projektioita. Toisin sanoen muuttujat $s$ ja $\theta$ ovat diskreettejä, jonka vuoksi on luontevaa käsitellä myös muuttujia $x$ ja $y$ diskreetteinä. Muuttujien $x$ ja $y$ arvot jaetaan käytännössä aina tasaisille väleille, jolloin muodostuu pikseleistä koostuva kuva-alue. Kuva-alueen aktiivisuusjakauma oletetaan paloittain vakioksi pikselikohtaisesti. Tällöin yhtälön (\ref{eqn:radon-muunnos}) integraali palautuu suoran $\Omega$ läpäisemien pikselien painotetuksi summaksi, jossa painokertoimet ovat suoran kulkemat matkat kussakin pikselissä.
\textcolor{red}{Projektion diskreetti muoto}\\
\textcolor{red}{Havaintomatriisi liian suuri muistiin tallentamiseen}

\subsection{Analyyttiset menetelmät}
\textcolor{red}{FBP}

\subsection{Iteratiiviset menetelmät}
\textcolor{red}{MLEM, OSEM}

\subsection{Virhelähteet kuvanmuodostuksessa}
\subsection{OMEGA}