\section{Pohdinta}
Aiemmin OMEGA-ohjelmisto tuki ainoastaan kuva-alueen rotaatioon perustuvaa menetelmää projektioiden laskemiseen sinogrammimuotoiselle SPECT-datalle. Kuva-alueen rotaatioon perustuva projektio saattaa osoittautua hankalaksi sovittaa uusien SPECT-laitteiden geometrioille, joten vaihtoehdoksi projektion laskemiseen kehitettiin sädepohjainen malli. \textcolor{red}{Sädepohjaisen projektorin hyödyt verrattuna rotaatiopohjaiseen}

Tässä diplomityössä toteutettu sädepohjainen malli osoittautui toimivaksi ainakin simuloidulla fantomidatalla. Kuitenkin verrattuna kuva-alueen rotaatioon perustuvaan menetelmään, sädepohjaiset menetelmät ovat hitaita, etenkin mallinnettaessa kollimaattoria tarkasti.

Mallia ei ole kuitenkaan syytä hylätä mahdollisen hitauden vuoksi, koska SSIM-arvojen ja keskivirheen perusteella mallia 3 käytettäessä säteiden määrällä ei ole merkittävää vaikutusta rekonstruktion laatuun. Vaikka mallin 3 ainoastaan approksimoi kollimaattorin vaikutusta, sillä tuotettu kuva osoittautui laadukkaimmaksi. Suurimpana syynä tälle on säteiden asettuminen kuva-alueeseen. Mallilla 3 lasketuilla säteillä otetaan todenmukaisimmin huomioon ne kuva-alueen vokselit, joista peräisin olevia gammafotoneja havaitaan tietyssä detektorin pikselissä. \textcolor{red}{Säteiden määritys jotenkin toisin, satunnaisuus, ikkunointi}

Säteiden määrän lisäksi rekonstruktion laatuun vaikuttavat lukemattomat muut parametrit ja etenkin mahdollinen \textit{a priori} -tieto aktiivisuusjakaumasta, jota tässä työssä ei käytetty lainkaan. Siten mallin kannalta säteiden lukumäärän kasvattamisen sijasta olisi mielekkäämpää tutkia rekonstruktion \textit{a priori} -malleja ja niiden vaikutusta lopputulokseen. Lisäksi seuraavan OMEGA-julkaisun (v2.0.0) on suunniteltu sisältävän myös SPECT-datan sädepohjaista rekonstruktiota nopeuttava näytönohjaimella ajettava malli. \textcolor{red}{Projektion jatkokehitys, mahdollinen neljäs malli}

Tulosten merkittävyyttä rajoittaa se, että mallin testaus tapahtui täysin simuloidulla datalla ja suppealla datajoukolla. Oikeanlaisen toiminnan varmistamiseksi testausta täytyisi suorittaa suurella, mieluusti kliinisestä kontekstista kerätyllä datajoukolla. Testauksen toteuttaminen olemassa olevalla datalla vaatii kuitenkin toisiolain mukaiset eettiset luvat ja uuden kliinisen datan mittaamisessa täytyisi kerätä suostumus erikseen jokaiselta kuvattavalta potilaalta.