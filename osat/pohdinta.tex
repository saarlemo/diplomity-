\section{Pohdinta}
Kollimaattorin mallinnus monipuolistaa projektion laskentaan käytettäviä menetelmiä SPECT-kuvan rekonstruktiossa. Kuitenkin verrattuna kuva-alueen rotaatioon perustuvaan menetelmään, sädepohjaiset menetelmät ovat hitaita, etenkin mallinnettaessa kollimaattoria tarkasti. Mallia ei ole kuitenkaan syytä hylätä sen hitauden vuoksi ennen GPU-laskennan toteuttamista. Seuraava looginen kehityskohde on juuri GPU-laskenta ja OpenCL-yhteensopivuus mallin nopeuttamiseksi. 

Tulosten merkittävyyttä rajoittaa se, että mallin testaus tapahtui täysin simuloidulla datalla ja suppealla datajoukolla. Oikeanlaisen toiminnan varmistamiseksi testausta täytyisi suorittaa suurella, mieluusti kliinisestä kontekstista kerätyllä datajoukolla. Testauksen toteuttaminen olemassa olevalla datalla vaatii kuitenkin toisiolain mukaiset eettiset luvat ja uuden kliinisen datan mittaamisessa täytyisi kerätä suostumus erikseen jokaiselta kuvattavalta potilaalta.

On selvää, että säteiden määrän kasvattaminen vaikuttaa suorasti laskenta-aikaan. Mallissa 3 erot neljän ja 64 säteen välillä ovat helposti nähtävissä kuvista, mutta 64 ja 144 säteen välillä eroavaisuuksia on vain vähän. Rekonstruktioiden ajallinen eroavaisuus on kuitenkin kahdesta minuutista viiteen minuuttiin on kuitenkin merkittävä, erityisesti kliinisessä kontekstissa. Kun huomioidaan mahdollinen GPU-laskennan tuoma parannus laskenta-ajassa, optimaalinen säteiden lukumäärä voisi olla 64 ja 144 säteen välillä. Huomioitavia kuvan laatuun ja laskenta-aikaan liittyviä muuttujia on kuitenkin lukemattomasti, jonka vuoksi optimaalista säteiden lukumäärää on mahdotonta määrittää universaalisti ja yksikäsitteisesti kaikkiin sovelluskohteisiin.