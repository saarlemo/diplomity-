\section{Pohdinta}
%Kollimaattorin mallinnus monipuolistaa projektion laskentaan käytettäviä menetelmiä SPECT-kuvan rekonstruktiossa. Kuitenkin verrattuna kuva-alueen rotaatioon perustuvaan menetelmään, sädepohjaiset menetelmät ovat hitaita, etenkin mallinnettaessa kollimaattoria tarkasti. Mallia ei ole kuitenkaan syytä hylätä sen hitauden vuoksi ennen GPU-laskennan toteuttamista. Seuraava looginen kehityskohde on juuri GPU-laskenta ja OpenCL-yhteensopivuus mallin nopeuttamiseksi. 

%Tulosten merkittävyyttä rajoittaa se, että mallin testaus tapahtui täysin simuloidulla datalla ja suppealla datajoukolla. Oikeanlaisen toiminnan varmistamiseksi testausta täytyisi suorittaa suurella, mieluusti kliinisestä kontekstista kerätyllä datajoukolla. Testauksen toteuttaminen olemassa olevalla datalla vaatii kuitenkin toisiolain mukaiset eettiset luvat ja uuden kliinisen datan mittaamisessa täytyisi kerätä suostumus erikseen jokaiselta kuvattavalta potilaalta.

%Työssä toteutettuja malleja ei myöskään analysoitu tehokkuuden suhteen. On selvää, että esimerkiksi säteiden määrän kasvattaminen vaikuttaa suorasti laskenta-aikaan, mutta kvantitatiivista dataa tehokkuudesta ei kerätty. Analysoimalla tehokkuutta voidaan selvittää esimerkiksi optimaalisia parametrejä mallin kliiniselle käytölle.

%Malli 3, jossa säteet jakautuvat neliön muotoisesti, on suunniteltu kuusikulmaista kollimaattorin reikää ajatellen. Neliön koko olisi suhteellisen suoraviivainen määrätä muuttujan avulla. Tällöin malli soveltuisi erityisen hyvin myös uusille SPECT/TT-laitteille, joissa kollimaattorin reikä on neliön muotoinen.