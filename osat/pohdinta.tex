\section{Pohdinta}
Aiemmin OMEGA-ohjelmisto tuki ainoastaan kuva-alueen rotaatioon perustuvaa menetelmää projektioiden laskemiseen sinogrammimuotoiselle SPECT-datalle. Kuva-alueen rotaatioon perustuva projektio saattaa osoittautua hankalaksi sovittaa uusien SPECT-laitteiden geometrioille, joten vaihtoehdoksi projektion laskemiseen kehitettiin sädepohjainen malli. Rotaatioon perustuvan projektion laskennassa täytyy myös interpoloida kuva-alueen vokseleiden arvoja, josta syntyy virhettä.

Tässä diplomityössä toteutetut sädepohjaiset mallit osoittautuivat toimiviksi ainakin fantomidatalla. Kuitenkin verrattuna kuva-alueen rotaatioon perustuvaan menetelmään, sädepohjaiset menetelmät ovat hitaita, etenkin mallinnettaessa kollimaattoria tarkasti.

Malleja ei ole kuitenkaan syytä hylätä mahdollisen hitauden vuoksi, koska todellisen SPECT-datan tapauksessa mallilla 3 laskettu projektio vastaa rotaatioon perustuvaa projektiota jo 16 säteellä SSIM-arvojen ja keskivirheen perusteella. Vaikka malli 3 ainoastaan approksimoi kollimaattorin vaikutusta, sillä tuotettu kuva osoittautui laadukkaimmaksi kolmen mallin kesken. Suurimpana syynä tälle on säteiden asettuminen kuva-alueeseen. Etenkin pienillä säteiden määrillä, mallilla 3 otetaan todenmukaisimmin huomioon ne kuva-alueen vokselit, joista peräisin olevia gammafotoneja havaitaan tietyssä detektorin pikselissä.

Toteutetuissa sädepohjaisissa malleissa säteet jakautuivat myös mahdollisimman tasaisesti kollimaattorin reiän sisälle. Pienillä säteiden määrillä kollimaattorin rajaaman kartion reunoilta peräisin olevia säteilytapahtumia painotetaan laskennassa yhtä paljon kuin kartion keskeltä peräisin olevia säteilytapahtumia. Tosiasiassa havaittujen säteilytapahtumien alkuperä on huomattavasti todennäköisemmin kartion keskellä kuin kartion reunoilla.

Tämän vaikutuksen voi ottaa huomioon usealla tavalla. Ensimmäinen keino on määrittää havaittujen säteilytapahtumien mahdollinen alkuperä tarkemmin ja huomioida se säteiden laskennassa. Tällöin kartion keskelle asetettaisiin enemmän säteitä kun sen reunoille.

Toinen keino on pitää säteiden laskenta ennallaan, mutta painottaa projektion laskennassa kartion keskelle asettuvia säteitä. Koska jokaiselle säteelle erikseen lasketut projektiot keskiarvoistetaan pikselikohtaisesti, muutos vaatisi ainoastaan keskiarvon muuttamisen painotetuksi. Painojen määräämiseen voisi soveltua kaksiulotteinen ikkunointi, joka huomioisi säteen sijainnin kollimaattorin reiän suhteen. Ikkunointiin on olemassa lukuisia valmiita toteutuksia yleisimmillä ohjelmointikielillä.

Säteilyn satunnainen luonne johtaa ajatukseen satunnaisuuden hyödyntämisestä projektion laskennassa. Kolmas keino tarkentaa sädepohjaista mallia voisi olla satunnaisuus säteen määrittämisessä. Käytännössä kartion sisältä valittaisiin kaksi pistettä satunnaisesti siten, että niiden virittämä säde lävistäisi detektorin pikslein sekä pysyisi kartion sisällä. Tämä menetelmä olisi suhteellisen suoraviivainen toteuttaa aiempien mallien pohjalta.

Säteiden määrän ja jakautumisen lisäksi rekonstruktion laatuun vaikuttavat lukemattomat muut parametrit ja etenkin mahdollinen \textit{a priori} -tieto aktiivisuusjakaumasta, jota tässä työssä ei käytetty lainkaan. Siten mallin kannalta säteiden lukumäärän kasvattamisen sijasta olisi mielekkäämpää tutkia rekonstruktion \textit{a priori} -malleja ja niiden vaikutusta lopputulokseen. Lisäksi seuraavan OMEGA-julkaisun (v2.0.0) on suunniteltu sisältävän myös SPECT-datan sädepohjaista rekonstruktiota nopeuttava näytönohjaimella ajettava malli.

Tulosten merkittävyyttä rajoittaa se, että mallin testaus tapahtui suppealla datajoukolla. Mallin oikeanlaisen toiminnan varmistamiseksi testausta täytyisi suorittaa suurella, mieluusti kliinisestä kontekstista kerätyllä, datajoukolla. Testauksen toteuttaminen olemassa olevalla datalla vaatii kuitenkin toisiolain mukaiset eettiset luvat ja uuden kliinisen datan mittaamisessa täytyisi kerätä suostumus erikseen jokaiselta kuvattavalta potilaalta.