\section{Yksifotoniemissiotomografia}
\textcolor{red}{Tähän yleisesti emissiotomografiasta}
\subsection{Säteilyfysiikka}
Radioaktiivinen hajoaminen on prosessi, jossa epävakaat atomiytimet muuttuvat vakaammiksi lähettäen säteilyä. Vakainta ytimen tilaa sanotaan perustilaksi. Virittyneessä ja metastabiilissa tilassa oleva ydin voi puolestaan spontaanisti muuttua toiseksi, vakaammaksi ytimeksi. Virittyneen ja metastabiilin tilan keskinäinen ero on spontaaniin hajoamiseen kuluva keskimääräinen aika: jos hajoamiseen kuluu yli \qty{1e-12}{\second}, atomia pidetään metastabiilina\cite{cherry_basic_2012}.

Hajoamisessa syntynyt säteily voi olla hiukkassäteilyä (alfa- tai beetasäteilyä) tai sähkömagneettista säteilyä (gammasäteilyä).\cite{cherry_basic_2012, cherry_interaction_2012} Luokittelu tehdään siksi, koska energian siirtymisen periaate ja vuorovaikutus aineen kanssa eroavat oleellisesti hiukkas- ja sähkömagneettisessa säteilyssä.

Hiukkassäteilyssä energia siirtyy liikkuvien hiukkasten kineettisenä energiana verrattain hitaasti, noin \qtyrange{5}{7}{\percent} valon nopeudesta. Hiukkasen verrattain suuren koon vuoksi hiukkassäteily ei ole läpitunkevaa. Esimerkiksi $\alpha$-hiukkanen pysähtyy paperiarkkiin ja $\beta$-hiukkasen pysäyttämiseen riittää akryylilevy.

Sähkömagneettisen säteilyn energia on sähkömagneettisessa kentässä, jossa informaatio siirtyy täsmälleen valon nopeudella.\cite{cherry_basic_2012} Sähkömagneettista säteilyä ovat esimerkiksi gammasäteily sekä näkyvä valo. Vaikka gammasäteily ja näkyvä valo ovat molemmat sähkömagneettista säteilyä, gammasäteily on merkittävästi vaarallisempaa. Ero johtuu yksittäisen kvantin energiasta, joka on tuhansia kertoja suurempi gammakvantilla kuin näkyvällä valolla. Radioaktiivisessa hajoamisessa syntyvä säteily on luonteeltaan ionisoivaa: vuorovaikuttaessaan aineen kanssa säteily voi irrottaa atomeista elektroneja\cite{cherry_interaction_2012}. Gammasäteily on erittäin läpitunkevaa korkeaenergisenä sähkömagneettisena säteilynä.

Hajoamislajeista yleisimmät ovat
\begin{itemize}
    \item $\beta^{-}$-hajoaminen, jossa neutroni muuttuu protoniksi, elektroniksi ja antineutriinoksi.  Hajoamisessa vapautuva energia siirtyy elektronin liike-energiaksi.
    \item $\beta^{+}$-hajoaminen, jossa protoni muuttuu neutroniksi, positroniksi ja neutriinoksi. Hajoamisessa vapautuva energia siirtyy positronin liike-energiaksi. Positroni vuorovaikuttaa aineen kanssa siirtäen liike-energiaa aineeseen. Lopulta positroni annihiloituu aineen elektronin kanssa muodostaen kaksi gammakvanttia.
    \item isomeerinen siirtymä (IT, \textit{isomeric transition}), jossa ydin siirtyy virittyneestä tilasta perustilaan emittoimalla gammakvantin.
    \item sisäinen konversio (IC, \textit{internal conversion}), jossa ydin siirtyy virittyneestä tilasta perustilaan siirtäen energian elektronille, joka poistuu atomista.
    \item elektronisieppaus, jossa ytimen protoni sieppaa ydintä kiertävän elektronin, muuttuen neutroniksi ja neutriinoksi. Hajoamisessa syntyvä ydin on mahdollisesti virittyneessä tilassa, joka voi purkautua joko isomeerisella siirtymällä tai sisäisellä konversiolla.
    \item $\alpha$-hajoaminen, jossa raskas ydin emittoi $\alpha$-hiukkasen, jossa on kaksi protonia ja kaksi neutronia.
    \item fissio, jossa raskas ydin hajoaa kahdeksi tai useammaksi kevyemmäksi ytimeksi, jolloin vapautuu energiaa ja useita neutroneja.
\end{itemize}

\textcolor{red}{Gammasäteilyn vuorovaikutus aineen kanssa}\\

\subsection{Merkkiaineet}
Atomien välisten sidosten muodostumisen määrittää pitkälti elektronien sijanti. Koska radioaktiivinen hajoaminen tapahtuu atomin ytimessä, virittyneessä tilassa oleva ydin ei vaikuta atomin kemiallisiin sidoksiin. Vastaavasti kemialliset sidokset eivät vaikuta atomin radioaktiivisuuteen.\cite{cherry_modes_2012} Juuri tämä mahdollistaa radioisotooppien käytön isotooppilääketieteessä, kun radioaktiivinen yhdiste käyttäytyy biologisesti täsmälleen samoin kuin vakaa yhdiste. Käytännössä biologisesti vaikuttavasta molekyylistä, eli merkkiaineesta, korvataan jokin atomi sen radioaktiivisella isotoopilla tai hyvin samankaltaisesti käyttäytyvällä molekyylillä. Esimerkiksi glukoosilla on elintärkeä rooli ihmisen aineenvaihdunnassa. Korvaamalla glukoosin toinen hydroksyyliryhmä radioaktiivisella \ce{^{18}F}-atomilla saadaan fluorodeoksyglukoosi (FDG). FDG vaikuttaa elimistössä tavallisen glukoosin tavoin, jolloin havaitsemalla radioaktiivisen fluorin hajoamistapahtumat ihmiskehon ulkopuolelta voidaan arvioida glukoosin jakautumista kehossa. \ce{^{18}F} on $\beta^{+}$-aktiivinen, josta johtuen sen käyttö kuvantamisessa painottuu positroniemissiotomografiaan. Yksifotoniemissiotomografiassa yleisimmin käytetty isotooppi on metastabiili \ce{^{99m}Tc}, jolla voidaan vastaavasti muodostaa eri tavalla vaikuttavia merkkiaineita.\textcolor{red}{Tähän lähde}

\textcolor{red}{Metastabiilien radionuklidien merkitys, suhteellisen pitkä puoliintumisaika $\Longrightarrow$ helppo erottaa muista radionuklideista ja puhdas gammalähde}


\subsection{Gammakamera}
\textcolor{red}{Valomonistinputki / puolijohdeilmaisin}\\
\textcolor{red}{Kollimaattori}\\
\textcolor{red}{Muun laitteiston toimintaperiaate pääpiirteittäin}