\section{Yksifotoniemissiotomografia}
\textcolor{red}{Tähän yleisesti emissiotomografiasta}
\subsection{Säteilyfysiikka}
Energiaa voi siirtyä usealla eri tavalla, joista yksi on säteily.

\textcolor{red}{Radioaktiivinen hajoaminen: säteilyn syntyminen}\\
Atomin ytimen nukleonit, eli protonit ja neutronit, voivat olla joko perustilassa, virittyneessä tilassa tai metastabiilissa tilassa\cite{cherry_basic_2012}. Perustila on kaikkein vakain nukleonien muodostelma. Virittyneessä ja metastabiilissa tilassa oleva ydin puolestaan voi spontaanisti muuttua toiseksi, vakaammaksi ytimeksi. Hajoamisprosessia sanotaan radioaktiiviseksi hajoamiseksi.\cite{cherry_modes_2012} Virittyneen ja metastabiilin tilan ero on spontaaniin hajoamiseen keskimäärin kuluva aika: jos hajoamiseen kuluu yli \qty{1e-12}{\second}, atomi luokitellaan metastabiiliksi\cite{cherry_basic_2012}.

\textcolor{red}{Protonien, neutronien, elektronien ja atomien massan muuttumien gammakvantin energiaksi}\\
\textcolor{red}{Hajoamislajit ja niiden merkitys isotooppilääketieteessä}
\begin{itemize}
    \item $\beta^{-}$
    \item $\beta^{+}$
    \item IT, IC
    \item Elektronisieppaus
    \item $\alpha$
\end{itemize}

\textcolor{red}{Hiukkassäteily, sähkömagneettinen säteily ja niiden merkitys isotooppilääketieteessä}
Säteily jaetaan usein kahteen luokkaan: hiukkassäteilyyn ja sähkömagneettinen säteilyyn.\cite{cherry_basic_2012, cherry_interaction_2012} Luokittelu tehdään siksi, koska energian siirtymisen tapa ja nopeus eroaa oleellisesti hiukkas- ja sähkömagneettisessa säteilyssä.

Hiukkassäteilyssä energia siirtyy liikkuvien hiukkasten kineettisenä energiana verrattain hitaasti, noin \qtyrange{5}{7}{\percent} valon nopeudesta. Sähkömagneettisen säteilyn energia on sähkömagneettisessa kentässä, jossa informaatio siirtyy täsmälleen valon nopeudella.\cite{cherry_basic_2012} Hiukkassäteilystä esimerkkinä toimii raskaiden atomien hajoamisessa syntyvät $\alpha$-hiukkaset ja sähkömagneettista säteilyä ovat esimerkiksi gammasäteily sekä näkyvä valo. Vaikka gammasäteily ja näkyvä valo ovat molemmat sähkömagneettista säteilyä, gammasäteily on merkittävästi vaarallisempaa. Ero johtuu yksittäisen kvantin energiasta, joka on tuhansia kertoja suurempi gammakvantilla kuin näkyvällä valolla. Gammasäteily on luonteeltaan ionisoivaa: vuorovaikuttaessaan aineen kanssa säteilyn energia voi siirtyä gammakvantilta atomin elektronille, joka irtoaa atomista luoden varatun ionin\cite{cherry_interaction_2012}. 

\textcolor{red}{Gammasäteilyn vuorovaikutus aineen kanssa}\\

\subsection{Merkkiaineet}
Atomien välisten sidosten muodostumisen määrittää pitkälti elektronien sijanti. Koska radioaktiivinen hajoaminen tapahtuu atomin ytimessä, virittyneessä tilassa oleva ydin ei vaikuta atomin kemiallisiin sidoksiin. Vastaavasti kemialliset sidokset eivät vaikuta atomin radioaktiivisuuteen.\cite{cherry_modes_2012} Juuri tämä mahdollistaa radioisotooppien käytön isotooppilääketieteessä, kun radioaktiivinen yhdiste käyttäytyy biologisesti täsmälleen samoin kuin vakaa yhdiste. Käytännössä biologisesti vaikuttavasta molekyylistä, eli merkkiaineesta, korvataan jokin atomi sen radioaktiivisella isotoopilla tai hyvin samankaltaisesti käyttäytyvällä molekyylillä. Esimerkiksi glukoosilla on elintärkeä rooli ihmisen aineenvaihdunnassa. Korvaamalla glukoosin toinen hydroksyyliryhmä radioaktiivisella \ce{^{18}F}-atomilla saadaan fluorodeoksyglukoosi (FDG). FDG vaikuttaa elimistössä tavallisen glukoosin tavoin, jolloin havaitsemalla radioaktiivisen fluorin hajoamistapahtumat ihmiskehon ulkopuolelta voidaan arvioida glukoosin jakautumista kehossa. \ce{^{18}F} on $\beta^{+}$-aktiivinen, josta johtuen sen käyttö kuvantamisessa painottuu positroniemissiotomografiaan. Yksifotoniemissiotomografiassa yleisimmin käytetty isotooppi on metastabiili \ce{^{99m}Tc}, jolla voidaan vastaavasti muodostaa eri tavalla vaikuttavia merkkiaineita.\textcolor{red}{Tähän lähde}

\textcolor{red}{Metastabiilien radionuklidien merkitys, suhteellisen pitkä puoliintumisaika $\Longrightarrow$ helppo erottaa muista radionuklideista ja puhdas gammalähde}


\subsection{Gammakamera}
\textcolor{red}{Valomonistinputki / puolijohdeilmaisin}\\
\textcolor{red}{Kollimaattori}\\
\textcolor{red}{Muun laitteiston toimintaperiaate pääpiirteittäin}