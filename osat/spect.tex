\section{Yksifotoniemissiotomografia}
Tietokonetomografiassa eri puolilta kuvattavaa kohdetta havaitusta säteilystä pyritään muodostamaan kolmiulotteinen kuva säteilyn vaimenemisesta. Emissiotomografiassa säteilyn lähde on kuvattavan kohteen sisällä ja menetelmällä pyritään selvittämään radioaktiivisuuden jakauma kuvattavassa kohteessa\cite{cherry_gamma_2012, van_audenhaege_review_2015}, esimerkiksi kliinisissä sovelluksissa radionuklidi on kiinnitetty johonkin biologisesti aktiiviseen yhdisteeseen\cite{cherry_single_2012, van_audenhaege_review_2015}.

Emissiotomografia jaetaan yleisesti yksifotoniemissiotomografiaan ja positroniemissiotomografiaan. Positroniemissiotomografiassa havaitaan positronin annihilaatiotapahtumassa syntyvät kaksi vastakkaisiin suuntiin etenevää gammakvanttia\cite{cherry_single_2012}. Yksifotoniemissiotomografian radionuklidit lähettävät hajotessaan yhden tai useamman gammakvantin, joiden etenemissuunnat eivät korreloi keskenään\cite{cherry_single_2012, van_audenhaege_review_2015}.

\subsection{Säteilyfysiikka}
Radioaktiivinen hajoaminen on prosessi, jossa epävakaat atomiytimet muuttuvat vakaammiksi lähettäen säteilyä. Vakainta ytimen tilaa sanotaan perustilaksi. Virittyneessä ja metastabiilissa tilassa oleva ydin voi puolestaan spontaanisti muuttua toiseksi, vakaammaksi ytimeksi. Virittyneen ja metastabiilin tilan keskinäinen ero on spontaaniin hajoamiseen kuluva puoliintumisaika. Yhden puoliintumisajan jälkeen keskimäärin puolet ytimistä on siirtynyt vakaampaan tilaan. Jos hajoamiseen kuluu yli \qty{1e-12}{\second}, atomia pidetään metastabiilina\cite{cherry_basic_2012}.

Hajoamisessa syntynyt säteily voi olla hiukkassäteilyä (alfa- tai beetasäteilyä) tai sähkömagneettista säteilyä (gammasäteilyä).\cite{cherry_basic_2012, cherry_interaction_2012} Luokittelu tehdään siksi, koska energian siirtymisen periaate ja vuorovaikutus aineen kanssa eroavat oleellisesti hiukkas- ja sähkömagneettisessa säteilyssä.

Hiukkassäteilyssä energia siirtyy liikkuvien hiukkasten kineettisenä energiana verrattain hitaasti, noin \qtyrange{5}{7}{\percent} valon nopeudesta. Hiukkasen verrattain suuren koon vuoksi hiukkassäteily ei ole läpitunkevaa. Esimerkiksi $\alpha$-hiukkanen pysähtyy paperiarkkiin ja $\beta$-hiukkasen pysäyttämiseen riittää akryylilevy.\cite{cherry_interaction_2012} Tästä johtuen hiukkassäteilyä sellaisenaan ei voida käyttää kuvantamisessa\cite{cherry_gamma_2012}.

Sähkömagneettisen säteilyn energia on sähkömagneettisessa kentässä, jossa informaatio siirtyy täsmälleen valon nopeudella.\cite{cherry_basic_2012} Sähkömagneettista säteilyä ovat esimerkiksi gammasäteily sekä näkyvä valo. Vaikka gammasäteily ja näkyvä valo ovat molemmat sähkömagneettista säteilyä, gammasäteily on merkittävästi vaarallisempaa. Ero johtuu yksittäisen kvantin energiasta, joka on tuhansia kertoja suurempi gammakvantilla kuin näkyvällä valolla. Radioaktiivisessa hajoamisessa syntyvä säteily on luonteeltaan ionisoivaa: vuorovaikuttaessaan aineen kanssa säteily voi irrottaa atomeista elektroneja\cite{cherry_interaction_2012}. Gammasäteily on erittäin läpitunkevaa korkeaenergisenä sähkömagneettisena säteilynä.

Hajoamislajeista yleisimmät ovat\cite{cherry_modes_2012}
\begin{itemize}
    \item $\beta^{-}$-hajoaminen, jossa neutroni muuttuu protoniksi, elektroniksi ja antineutriinoksi.  Hajoamisessa vapautuva energia siirtyy elektronin liike-energiaksi.
    \item $\beta^{+}$-hajoaminen, jossa protoni muuttuu neutroniksi, positroniksi ja neutriinoksi. Hajoamisessa vapautuva energia siirtyy positronin liike-energiaksi. Positroni vuorovaikuttaa aineen kanssa siirtäen liike-energiaa aineeseen. Lopulta positroni annihiloituu aineen elektronin kanssa muodostaen kaksi gammakvanttia.
    \item isomeerinen siirtymä (IT, \textit{isomeric transition}), jossa ydin siirtyy virittyneestä tilasta perustilaan emittoimalla gammakvantin.
    \item sisäinen konversio (IC, \textit{internal conversion}), jossa ydin siirtyy virittyneestä tilasta perustilaan siirtäen energian elektronille, joka poistuu atomista.
    \item elektronisieppaus, jossa ytimen protoni sieppaa ydintä kiertävän elektronin, muuttuen neutroniksi ja neutriinoksi. Hajoamisessa syntyvä ydin on mahdollisesti virittyneessä tilassa, joka voi purkautua joko isomeerisella siirtymällä tai sisäisellä konversiolla.
    \item $\alpha$-hajoaminen, jossa raskas ydin emittoi $\alpha$-hiukkasen, jossa on kaksi protonia ja kaksi neutronia.
    \item fissio, jossa raskas ydin hajoaa kahdeksi tai useammaksi kevyemmäksi ytimeksi, jolloin vapautuu energiaa ja useita neutroneja.
\end{itemize}
Emissiotomografiassa tärkeimmät hajoamislajit ovat $\beta^{+}$, isomeerinen siirtymä, sisäinen konversio ja elektronisieppaus\cite{cherry_modes_2012}.

Gammasäteilyllä ja aineella on neljä vuorovaikutustapaa: valosähköinen ilmiö, Comptonin sironta, parinmuodostus ja Rayleighin sironta. Gammasäteilyn vuorovaikutusmekanismit ovat olennaisia isotooppilääketieteessä, koska ne määrittävät, kuinka säteily siirtää energiaa kudoksiin ja kuinka sitä voidaan käyttää diagnostiikassa ja hoidossa\cite{cherry_interaction_2012}. Esimerkiksi gammakamerat perustuvat valosähköiseen ilmiöön ja Comptonin sirontaan. Parinmuodostuksella ja Rayleighin sironnalla ei ole yleisesti käytössä olevia sovelluskohteita isotooppilääketieteessä, mutta parinmuodostus on tärkeä mekanismi korkean energian säteilyfysiikassa ja Rayleighin sironta täytyy huomioida tietokonetomografiassa.\cite{cherry_interaction_2012}

Valosähköisessä ilmiössä gammakvantti siirtää koko energiansa atomin elektronille, jolloin elektroni poistuu atomista. Valosähköinen ilmiö tapahtuu yleensä pienillä gammakvantin energioilla ($<$\qty{0.5}{\kilo\electronvolt}). Osa energiasta menee elektronin irrottamiseen atomista ja loput muuttuu kineettiseksi energiaksi.\cite{cherry_interaction_2012}

Comptonin sironnassa gammakvantti törmää atomissa olevaan heikosti sidottuun elektroniin luovuttaen osan energiastaan elektronille. Törmäyksen seurauksena elektroni irtoaa, mutta valosähköisestä ilmiöstä poiketen gammakvantti ei katoa. Gammakvantti jatkaa etenemistä pienemmällä energialla ja pidemmällä aallonpituudella. Comptonin sironta hallitsee energia-alueella \qtyrange{0.5}{3}{\mega\electronvolt}.\cite{cherry_interaction_2012}

Gammakvantti voi myös muuttua positroniksi ja elektroniksi törmätessään atomin ytimeen. Tämä vaatii gammakvantilta ainakin elektronin ja positronin yhteenlasketun lepomassan verran energiaa ekvivalenssiperiaatteen $E=mc^2$ mukaisesti, eli yli \qty{1.022}{\mega\electronvolt}. Kaavassa $E$ on energia, $m$ on massa ja $c$ on valonnopeus. Jäljelle jäävä energia jakautuu positronin ja elektronin kineettisiin energioihin liikemäärän säilymislain mukaisesti eli positroni ja elektroni etenevät vastakkaisiin suuntiin.\cite{cherry_interaction_2012}

Rayleighin sironnassa gammakvantin ja atomin voidaan ajatella törmäävän toisiinsa elastisesti. Verrattaessa Comptonin sironnassa irtoavaan elektroniin, atomin massa on hyvin suuri eli liike-energiasta siirtyy atomille vain hyvin pieni osa. Tästä johtuen Rayleighin sironnalla ei ole käytännön merkitystä isotooppilääketieteessä. Rayleighin sironta on myös merkittävää ainoastaan matalaenergisillä (alle \qty{50}{\kilo\electronvolt}) gammakvanteilla.\cite{cherry_interaction_2012}

\subsection{Merkkiaineet}
Atomien välisten sidosten muodostumisen määrittää pitkälti elektronien sijanti. Koska radioaktiivinen hajoaminen tapahtuu atomin ytimessä, virittyneessä tilassa oleva ydin ei vaikuta atomin kemiallisiin sidoksiin. Vastaavasti kemialliset sidokset eivät vaikuta atomin radioaktiivisuuteen.\cite{cherry_modes_2012} Juuri tämä mahdollistaa radioisotooppien käytön isotooppilääketieteessä, kun radioaktiivinen yhdiste käyttäytyy biologisesti täsmälleen samoin kuin vakaa yhdiste. Käytännössä biologisesti vaikuttavasta molekyylistä, eli merkkiaineesta, korvataan jokin atomi sen radioaktiivisella isotoopilla tai hyvin samankaltaisesti käyttäytyvällä molekyylillä\cite{cherry_modes_2012, crisan_radiopharmaceuticals_2022}. Esimerkiksi glukoosilla on elintärkeä rooli ihmisen aineenvaihdunnassa. Korvaamalla glukoosin toinen hydroksyyliryhmä $\beta^{+}$-aktiivisella \ce{^{18}F}-atomilla saadaan fluorodeoksyglukoosi (FDG). FDG vaikuttaa elimistössä tavallisen glukoosin tavoin, joten havaitsemalla positronien annihilaatiotapahtumista syntyvät gammakvantit ihmiskehon ulkopuolelta voidaan arvioida glukoosin jakautumista kehossa.\cite{crisan_radiopharmaceuticals_2022}

Yksifotoniemissiotomografiassa yleisimmin käytetty isotooppi on metastabiili \ce{^{99m}Tc}, josta voidaan vastaavasti muodostaa eri tavalla vaikuttavia merkkiaineita. \ce{^{99m}Tc}:n puoliintumisaika on \qty{6}{\hour} ja hajoamisessa syntyvän gammakvantin energia on noin \qty{141}{\kilo\electronvolt}.\cite{cherry_interaction_2012, cherry_single_2012, crisan_radiopharmaceuticals_2022} Metastabiilin technetiumin ominaisuudet tekevät siitä lähes optimaalisen gammakameralla havaitsemiseen. Tämän lisäksi, koska \ce{^{99m}Tc} muodostaa hajotessaan ainoastaan gammasäteilyä ja puoliintumisaika on suhteellisen lyhyt, kuvantamisen säteilyaltistus pysyy pienenä.\cite{cherry_modes_2012, crisan_radiopharmaceuticals_2022}

Muita yksifotoniemissiotomografiassa yleisesti käytettyjä radioisotooppeja ovat jodi-123, xenon-133, tallium-201 ja indium-111\cite{crisan_radiopharmaceuticals_2022}.

\subsection{Gammakamera}
Gammakuvantamisessa havaittava säteily on energialtaan \qtyrange{80}{500}{\kilo\electronvolt}. Tämän energia-alueen fotonit lävistävät ihmiskehon tehokkaasti, mutta ovat helposti pysäytettävissä tiheällä tuikeaineella ja lyijyllä.\cite{cherry_gamma_2012} Yksifotoniemissiotomografiassa käytetyn gammakameran rakenteen poikkileikkaus on esitetty \hyperref[fig:spect-detektori]{Kuvassa \ref*{fig:spect-detektori}}. Kamera koostuu valon elektroneiksi muuttavista valomonistinputkista, joiden päällä on levy säteilyn näkyväksi valoksi muuttavaa tuikeainetta. Näiden päällä on säteilyä valikoivasti läpäisevä kollimaattori.\cite{van_audenhaege_review_2015, cherry_gamma_2012}

\begin{figure}[H]
    \centering
    \captionsetup{width=.9\textwidth}
    \begin{tikzpicture}
    % Säteily
    \foreach \x in {-0.1, -0.06, ..., 0.14} {
        \draw[red] (3.5 + \x, 0) -- (3.5 - 2*4.5*\x, 5);
    }
    \node[right] at (8, 3) {Säteily};

    % Kollimaattori
    \foreach \x in {0,0.3,...,7}
        \fill (\x,0) rectangle (\x + 0.1,1);
    \node[right] at (8, 0.5) {Kollimaattori};

    % Tuikeaine
    \draw (0,0) rectangle (7,-1);
    \fill[pattern=north east lines] (0,0) rectangle (7,-1);
    \node[right] at (8, -0.5) {Tuikeaine};

    % Valomonistinputket
    \foreach \x in {0.5,1.5,...,6.5}{
        \fill[blue!40] 
        (\x-0.5,-1) --
        (\x+0.5,-1) --
        (\x+0.3,-1.5) --
        (\x+0.3,-3) --
        (\x-0.3,-3) --
        (\x-0.3,-1.5) --
        cycle;
    }
    \node[right] at (8, -2) {Valomonistinputket};
\end{tikzpicture}
    \caption{Yksifotoniemissiotomografiassa käytetyn gammakameran rakenteen poikkileikkaus. Kuvassa esitetyt gammakameran osat ovat ylhäältä alaspäin kollimaattori, joka päästää läpi vain tietystä suunnasta tulevan säteilyn, tuikeaine, joka muuttaa gammasäteilyn näkyväksi valoksi sekä valomonistinputket, jotka muuttavat näkyvän valon elektroneiksi. Punaisella värillä on havainnollistettu aluetta, jolta yksi kollimaattorin reikä kerää säteilyä detektorille.}
    \label{fig:spect-detektori}
\end{figure}

Tärkein gammakameran osa kuvan laadun kannalta on kollimaattori, joka sijaitsee säteilylähteen ja tuikeainekiteen välissä. Kollimaattorin tarkoituksena on päästää detektorille vain tietystä suunnasta tuleva säteily ja vaimentaa muista suunnista tuleva säteily. Nämä toivotut ominaisuudet toteuttaa parhaiten raskaasta alkuaineesta valmistettu reikälevy, jossa reiät ovat yleensä kuusikulmaisia, pyöreitä tai neliön muotoisia.\cite{van_audenhaege_review_2015, cherry_gamma_2012} Mahdollisia materiaaleja reikien välille ovat muun muassa lyijy, wolframi, kulta, uraani ja platina\cite{van_audenhaege_review_2015}, mutta kustannusten vuoksi\cite{van_audenhaege_review_2015} kollimaattori on lähes aina valmistettu lyijystä tai wolframista.\cite{cherry_gamma_2012}

\textcolor{red}{Kollimaattori: eri tyypit}\\

\textcolor{red}{Tuikeilmaisin}\\
\textcolor{red}{Valomonistinputki}\\
\textcolor{red}{Puolijohdeilmaisin}\\
\textcolor{red}{Muun laitteiston toimintaperiaate pääpiirteittäin}