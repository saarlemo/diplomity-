\begin{titlepage}
    \section*{Tiivistelmä}
    %Yksifotoniemissiotomografia (SPECT, \textit{single-photon emission tomography}) on lääketieteellinen kuvantamismenetelmä, jonka tarkoituksena on määrittää gamma-aktiivisen merkkiaineen aktiivisuusjakauma kehossa. Kuvantamismenetelmää käytetään esimerkiksi sydän- ja verisuonitautien, aivohalvauksen, infektioiden ja syövän diagnosoinnissa. Tietokonetomografiasta (TT) poiketen SPECT soveltuu elinten ja kudosten fysiologisen toiminnan seuraamiseen eikä pelkästään anatomisen rakenteen kuvantamiseen. Keskeinen osa SPECT-laitteiston suorituskyvyn taustalla on kollimaattori, jonka tehtävänä on suodattaa detektorilla havaitut gammafotonit etenemissuunnan perusteella. Koska kollimaattori vaikuttaa merkittävästi järjestelmän resoluutioon ja herkkyyteen, sen tarkka mallintaminen on olennaista ottaa huomioon kuvan rekonstruktioprosessissa.

    %Tässä diplomityössä toteutetaan SPECT-rekonstruktio alkujaan positroniemissiotomografiakuvan (PET) rekonstruktiota varten suunniteltuun avoimen lähdekoodin OMEGA-ohjelmistoon. Ensisijainen tavoite oli toteuttaa kollimaattorin mallinnusalgoritmi, joka voidaan suorittaa tietokoneen keskusyksiköllä (CPU) ja joka on yhteensopiva SPECT-laitteiden kuusikulmaisten ja neliönmuotoisten kollimaattorin reikien kanssa. Toteutetussa menetelmässä olennaisinta on gammafotonien kulkureittien määrittäminen kolmiulotteisessa avaruudessa kollimaattorin reikien muodon ja sijainnin perusteella. Toteutetulla SPECT-kuvan rekonstruktiolla voidaan sellaisenaan hyödyntää OMEGAan sisällytettyjä yleisimpiä rekonstruktiomenetelmiä ja -algoritmeja.

    %Mallin testaaminen tapahtui simuloidulla datalla. Saadut tulokset olivat lupaavia, mutta mallin luotettavuuden varmistamiseksi tarvitaan lisävalidointia laajemmalla, kliinisellä aineistolla. Malli havaittiin myös verrattain hitaaksi, mutta optimaalisten parametrien valintaan tarvitaan kvantitatiivista analyysia mallin laskentatehokkuudesta. Toinen mahdollinen SPECT-rekonstruktion kehityssuunta on näytönohjaimella tapahtuva laskenta, joka nopeuttaisi laskentaa merkittävästi.
\end{titlepage}